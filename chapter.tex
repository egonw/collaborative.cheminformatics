\documentclass[12pt]{book}
\usepackage[utf8]{inputenc}
\usepackage{a4wide}
\usepackage{amsmath}
\usepackage{hyperref}
\usepackage{times}

\begin{document}

\chapter{Collaborative Cheminformatics Applications}

\section{Background}

\section{Collaborative Cheminformatics Code Development}

JChemPaint was an early project promoting collaborative development over the
internet, going back to 1998~\cite{Krause2000}.

The Blue Obelisk movement was founded in 2005 to jointly promote Open Data,
Open Source, and Open Standards... \cite{Guha2006}.

Source code repositories: SourceForge, GitHub, ...

\subsection{Peer review}

GitHub provides functionality to peer review commits ...

% TODO: copy/paste screenshot from Egon's blog

\section{Collaborative Cheminformatics Knowledge Base Building}

Open Data ...

The Blue Obelisk Data Repository is a collaborative project by a number
of cheminformatics tools, including Kalzium (of the KDE desktop), the
Chemistry Development Kit, ...

% TODO: include screenshot of Kalzium

Social sites to share Open Data include the
NMRShiftDB~\cite{Steinbeck2004} (GNU FDL license),
ChemPedia (\url{http://chempedia.org/substances}, CC0 license)...

% TODO: perhaps move these resources into a table

\subsection{Linking Knowledge Bases}

Key to collaboration is also the sharing of knowledge. A prominent role here
plays the linking of various databases, which allows integration of them.

Resource Description Framework ... HCLS, Bio2RDF, Chem2Bio2RDF, ...

Userscripts ...

\section{Collaborative Computing}

The development of infrastructure and tools to link knowledge bases is
a fundamental requirement for efficient colaborations. The previous
seections have highlighted a variety of efforts in these areas. While
it is true that collaborative efforts (in any field) are primarily a
function of social interactions, it is important to remember that
collaborations need not be directly between individuals. Rather, they
can also be mediated by software. From this point of view, there have
been a number of developments in the last few years that allow
individuals, unrelated to each other in terms of formal collaborative
aggreements, to interact with each others' resources. But such
interactions do not necessarily have to involve remote
resources. Instead, a collaboration could also be in the form of
shared specifications.That is, individuals could collaborate on the
specification of a process or program, which would then be run locally
using each collaborators own resources. Finally, to achieve these
types of collaborative efforts, technologies to support these are
necessary; many of these are well established such as mailing lists
and chat systems, whereas a number are more recent such as service
registries. The following subsections discuss these facets of
collaborative computing in more detail.

\subsection{Shared Computing Services}

Distributed computing has gone through many phases starting with
Remote Procedure Calls (RPC) in the 1970a and 1980s to CORBA in the
1990s. More recently, distributed computing in the form of web
services have become a mechanism by which different groups expose both
methods (analogous to remote function calls that are the hallmark of
RPC and CORBA) as well as data. In this sense, we consider web
services to be a generalization, allowing one to provide access both
data (such as databases) and algorithms.

There are many protocols by which web services such as SOAP and
XMPP~\cite{Wagener2009}. Most web service protocols are designed to
work with multiple types of transport layers (HTTP, UDP, etc) but the
majority of web service protocols currently in use work over
HTTP. This approach results in significant;y easier deploy of services
and access to them, since HTTP is ubiquitous throughout the
Internet. For a more detailed review of web service technologies the
reader is referred to XXX and YYY.

In this section we provide a brief overview of cheminformatics web
services and some use cases highlighting the collaborative potential
of such web services.

Indiana University has developed a number of cheminformatics web
services [REF] that provide access to core cheminformatics methods
(fingerprints, 2D depiction and various molecular descriptors),
statistical techniques (using R [REF] as the backend) and chemical
database access methods. Most of these services are implemented in
Java using the Chemistry Development Kit [REF] and SOAP as the
underlying protocol. A detailed description of these services can be
found in Dong et al~\cite{Dong2007Web}. The services have been used in
a variety of scenarios. For example, the fingerprint and statistical
model services were employed by Indiana University to develop
models to predict the activity of user supplied compounds against the
NCI60 cancer cell lines. Interestingly, the final predictive model was
itself, converted to a web service and thus accessible by any SOAP client,
remote or local. Note that while the model service was also located at
Indiana University, it could easily make use of fingerprint services
located at other sites, anywhere across the world.

Recently, these web services have been forked and reimplemented as
REST based services. While SOAP services can be ``hidden'' behind a
REST interface, the reimplementation avoids the extra complexity
imposed by SOAP, by directly exposing the functionality via the REST
interface. These services can be found at
\href{http://rguha.net/rest}{http://rguha.net/rest}. Curently, the
services are hosted at multiple locations include Drexel University,
USA and Uppsala University, Sweden and are utilized by a number of
independent applications. An example is the use of these services to
enhance an Open Notebook Science (ONS) project. The ONS Solubility
Challenge is an ongoing project in which a number of research groups
have experimentally determined the solubility of a variety of solutes
in a variety of solvents. All the data generated as part of this
project is publicaly hosted on a GoogleDocs spreadsheet and outsiders
are encouraged to explore and mine the data, with the hope that their
results will be also be made Open. At the time of writing the project
has seen contributions from a number of people, including chemists,
mathematicians and programmers. As the project has grown, the number
of measurements now numbers in the hundreds. The spreadsheet contains
alphanumeric identifiers for solutes and solvents along with SMILES
representations and solubility data. In a number of cases, external
references are also included. While the use of Google Spreadsheets is
a very simple way to share data, the nature of the data makes it
unweildy to explore. In general, while numeric solubility data is
useful it is more appropriate to explore it from a chemical point of
view - that is, in terms of structures and substructures.

As a result, a simple web page interface
(\href{http://toposome.chemistry.drexel.edu/~rguha/jcsol/sol.html}{http://toposome.chemistry.drexel.edu/~rguha/jcsol/sol.html})
was developed that extract data from the Google spreadsheet via the
Google-provided Data API and presented the data or filtered subsets of
the data (based on solute or solvent identifiers, substructures or
solubility ranges). The key feature of this application is the
incorporation of chemical intelligence, by making use of
cheminformatics web services hosted at Uppsala University. By making
use of these services, the SMILES strings stored in the spreadsheet
could be filtered by the presence or absence of substructures
(specified via SMARTS). In addition, the services were also employed
to provide 2D structure depictions of the results matching satisfying
the query. From a colaborative point of view, this application is
interesting as the developer had no role in the gathering of the
solubility data and did not create the online spreadsheet. Instead,
the application made use of public data API's provided by Google and
public web services hosted at another, remote location to extract and
present data that satisfied the requirements of another, external
group (i.e., the experimentalists making the measurements).

A related application (XXX) was developed by another researcher to
explore the chemical space (via descriptor calculations followed by
principal components analysis). This application also made use of the
cheminformatics services hosted at Drexel University as well as
visualization services provided by Google, allowing users to generate
principal component plots of the solubility data, thereby
understanding the extent of the chemical space occupied by the current
set of chemical.

These applications highlight the fact that distributed software
resources were key in allowing multiple, unrelated parties to
collaborate on a publically available dataset. While this type of
collaboration could certainly be achieved using traditional software
resources (i.e., locally installed libraries and programs), the
presence of freely accessible web services (for both cheminformatics
as well as data and visualization) allows \emph{arbitrary} individuals
or groups to develop novel applications that were not considered by
the original researchers. Furthermore, the free and distributed nature
of the resources allows such developers free themselves of software
installation and management and onerous licensing conditions.


\subsection{Sharing Computation Specifications}

A simple example of a ``computation specification'' is the source code
of a program or even and input file for a program. In both cases, the
document fully specifies what is required to run a program to achieve
a desired result. Clearly there have been many mechanisms to share
such specifications. However, with a few exceptions these sharing
mechanisms tend to simply collect source code, for example, in a
central repository and let users access it from there. Usually, there
is no meta data attached or associated with the source code and thus
exploring related cases is difficult.

Recently, there has been significant amounts of efforts devoted to the
development of ``workflow'' or ``pipeline'' tools for chem- and
bioinformatics. These tools encapsulate core cheminformatics tasks
such as reading in SMILES, evaluating descriptors and so on in simple
graphical elements (usually boxes and connectors). A user can then
arrange sequences of such elements to perform a task. The GUI approach
coupled with the fact that in most cases, such workflow tools require
no programming knowledge make such tools very attractive to
experimentalists and other users with limited programming
experience. A number of such tools are available including Pipeline
Pilot, KNIME and Taverna.

A recent development around workflow tools is the sharing of workflow
specifications (or ``programs''), exemplified by the MyExperiment
website~\cite{Goble2010}.


\subsection{Collaboration Tools}

Taverna~\cite{Oinn2004} ... CDK-Taverna~\cite{Kuhn2010} ...

Bioclipse~\cite{Spjuth2009,Spjuth2007}...

\section{Managing Collaborative Projects in Cheminformatics}

Facilitating communication ... mailing lists, blogs, ...

Collaborative documenting ... Google Docs, Google Wave~\cite{Neylon2009}, ...

\section{Conclusion}

\bibliographystyle{unsrt}
\bibliography{chapter}

\end{document}


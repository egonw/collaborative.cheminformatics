\documentclass[12pt]{book}
\usepackage[utf8]{inputenc}
\usepackage{a4wide}
\usepackage{amsmath}
\usepackage{hyperref}
\usepackage{times}

\begin{document}

\chapter{Collaborative Cheminformatics Applications}

\section{Background}

\section{Collaborative Cheminformatics Code Development}

JChemPaint was an early project promoting collaborative development over the
internet, going back to 1998~\cite{Krause2000}.

The Blue Obelisk movement was founded in 2005 to jointly promote Open Data,
Open Source, and Open Standards... \cite{Guha2006}.

Source code repositories: SourceForge, GitHub, ...

\subsection{Peer review}

GitHub provides functionality to peer review commits ...

% TODO: copy/paste screenshot from Egon's blog

\section{Collaborative Cheminformatics Knowledge Base Building}

Open Data ...

The Blue Obelisk Data Repository is a collaborative project by a number
of cheminformatics tools, including Kalzium (of the KDE desktop), the
Chemistry Development Kit, ...

% TODO: include screenshot of Kalzium

Social sites to share Open Data include the
NMRShiftDB~\cite{Steinbeck2004} (GNU FDL license),
ChemPedia (\url{http://chempedia.org/substances}, CC0 license)...

% TODO: perhaps move these resources into a table

\subsection{Linking Knowledge Bases}

Key to collaboration is also the sharing of knowledge. A prominent role here
plays the linking of various databases, which allows integration of them.

Resource Description Framework ... HCLS, Bio2RDF, Chem2Bio2RDF, ...

Userscripts ...

\section{Collaborative Computing}

The development of infrastructure and tools to link knowledge bases is
a fundamental requirement for efficient colaborations. The previous
seections have highlighted a variety of efforts in these areas. While
it is true that collaborative efforts (in any field) are primarily a
function of social interactions, it is important to remember that
collaborations need not be directly between individuals. Rather, they
can also be mediated by software. From this point of view, there have
been a number of developments in the last few years that allow
individuals, unrelated to each other in terms of formal collaborative
aggreements, to interact with each others' resources. But such
interactions do not necessarily have to involve remote
resources. Instead, a collaboration could also be in the form of
shared specifications.That is, individuals could collaborate on the
specification of a process or program, which would then be run locally
using each collaborators own resources. Finally, to achieve these
types of collaborative efforts, technologies to support these are
necessary; many of these are well established such as mailing lists
and chat systems, whereas a number are more recent such as service
registries. The following subsections discuss these facets of
collaborative computing in more detail.

\subsection{Shared Computing Services}

Distributed computing has gone through many phases starting with
Remote Procedure Calls (RPC) in the 1970a and 1980s to CORBA in the
1990s. More recently, distributed computing in the form of web
services have become a mechanism by which different groups expose both
methods (analogous to remote function calls that are the hallmark of
RPC and CORBA) as well as data. In this sense, we consider web
services to be a generalization, allowing one to provide access both
data (such as databases) and algorithms.

There are many protocols by which web services such as SOAP and
XMPP~\cite{Wagener2009}. Most web service protocols are designed to
work with multiple types of transport layers (HTTP, UDP, etc) but the
majority of web service protocols currently in use work over
HTTP. This approach results in significant;y easier deploy of services
and access to them, since HTTP is ubiquitous throughout the
Internet. For a more detailed review of web service technologies the
reader is referred to XXX and YYY.

In this section we provide a brief overview of cheminformatics web
services and some use cases highlighting the collaborative potential
of such web services.

Indiana University has developed a number of cheminformatics web
services [REF] that provide access to core cheminformatics methods
(fingerprints, 2D depiction and various molecular descriptors),
statistical techniques (using R [REF] as the backend) and chemical
database access methods. Most of these services are implemented in
Java using the Chemistry Development Kit [REF] and SOAP as the
underlying protocol. A detailed description of these services can be
found in Dong et al~\cite{Dong2007Web}. The services have been used in
a variety of scenarios. For example, the fingerprint and statistical
model services were employed by Indiana University to develop
models to predict the activity of user supplied compounds against the
NCI60 cancer cell lines.




\subsection{Sharing Computation Specifications}

Computation workflows can be shared with MyExperiment.org~\cite{Goble2010}.

\subsection{Collaboration Tools}

Taverna~\cite{Oinn2004} ... CDK-Taverna~\cite{Kuhn2010} ...

Bioclipse~\cite{Spjuth2009,Spjuth2007}...

\section{Collaborative Projects in Cheminformatics}

Facilitating communication ... mailing lists, blogs, ...

Collaborative documenting ... Google Docs, Google Wave~\cite{Neylon2009}, ...

\section{Conclusion}

\bibliographystyle{unsrt}
\bibliography{chapter}

\end{document}


\documentclass[12pt]{book}
\usepackage[utf8]{inputenc}
\usepackage{a4wide}
\usepackage{amsmath}
\usepackage{hyperref}

\begin{document}

\chapter{Collaborative Cheminformatics Applications}

\section{Background}

\section{Collaborative Cheminformatics Code Development}

JChemPaint was an early project promoting collaborative development over the
internet, going back to 1998~\cite{Krause2000}.

The Blue Obelisk movement was founded in 2005 to jointly promote Open Data,
Open Source, and Open Standards... \cite{Guha2006}.

Source code repositories: SourceForge, GitHub, ...

\subsection{Peer review}

GitHub provides functionality to peer review commits ...

% TODO: copy/paste screenshot from Egon's blog

\section{Collaborative Cheminformatics Knowledge Base Building}

Open Data ...

The Blue Obelisk Data Repository is a collaborative project by a number
of cheminformatics tools, including Kalzium (of the KDE desktop), the
Chemistry Development Kit, ...

% TODO: include screenshot of Kalzium

Social sites to share Open Data include the
NMRShiftDB~\cite{Steinbeck2004} (GNU FDL license),
ChemPedia (\url{http://chempedia.org/substances}, CC0 license)...

% TODO: perhaps move these resources into a table

\subsection{Linking Knowledge Bases}

Key to collaboration is also the sharing of knowledge. A prominent role here
plays the linking of various databases, which allows integration of them.

Resource Description Framework ... HCLS, Bio2RDF, Chem2Bio2RDF, ...

Userscripts ...

\section{Collaborative Computing}

\subsection{Shared Computing Services}

Sharing computing with web services ... SOAP, REST, XMPP~\cite{Wagener2009}.

\subsection{Sharing Computation Specifications}

Computation workflows can be shared with MyExperiment.org~\cite{Goble2010}.

\subsection{Collaboration Tools}

Taverna~\cite{Oinn2004} ... CDK-Taverna~\cite{Kuhn2010} ...

Bioclipse~\cite{Spjuth2009,Spjuth2007}...

\section{Collaborative Projects in Cheminformatics}

Facilitating communication ... mailing lists, blogs, ...

Collaborative documenting ... Google Docs, Google Wave, ...

\section{Conclusion}

\bibliographystyle{unsrt}
\bibliography{chapter}

\end{document}


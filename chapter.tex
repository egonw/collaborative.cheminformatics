\documentclass[12pt]{book}
\usepackage[utf8]{inputenc}
\usepackage{a4wide}
\usepackage{amsmath}
\usepackage{hyperref}
\usepackage{times}

\begin{document}

\chapter{Collaborative Cheminformatics Applications}

\section{Background}

\section{Collaborative Cheminformatics Code Development}

JChemPaint was an early project promoting collaborative development over the
internet, going back to 1998~\cite{Krause2000}.

The Blue Obelisk movement was founded in 2005 to jointly promote Open Data,
Open Source, and Open Standards... \cite{Guha2006}.

Source code repositories: SourceForge, GitHub, ...

\subsection{Peer review}

GitHub provides functionality to peer review commits ...

% TODO: copy/paste screenshot from Egon's blog

\section{Collaborative Cheminformatics Knowledge Base Building}

Open Data ...

The Blue Obelisk Data Repository is a collaborative project by a number
of cheminformatics tools, including Kalzium (of the KDE desktop), the
Chemistry Development Kit, ...

% TODO: include screenshot of Kalzium

Social sites to share Open Data include the
NMRShiftDB~\cite{Steinbeck2004} (GNU FDL license),
ChemPedia (\url{http://chempedia.org/substances}, CC0 license)...

% TODO: perhaps move these resources into a table

\subsection{Linking Knowledge Bases}

Key to collaboration is also the sharing of knowledge. A prominent role here
plays the linking of various databases, which allows integration of them.

Resource Description Framework ... HCLS, Bio2RDF, Chem2Bio2RDF, ...

Userscripts ...

\section{Collaborative Computing}

The development of infrastructure and tools to link knowledge bases is
a fundamental requirement for efficient colaborations. The previous
seections have highlighted a variety of efforts in these areas. While
it is true that collaborative efforts (in any field) are primarily a
function of social interactions, it is important to remember that
collaborations need not be directly between individuals. Rather, they
can also be mediated by software. From this point of view, there have
been a number of developments in the last few years that allow
individuals, unrelated to each other in terms of formal collaborative
aggreements, to interact with each others' resources. But such
interactions do not necessarily have to involve remote
resources. Instead, a collaboration could also be in the form of
shared specifications.That is, individuals could collaborate on the
specification of a process or program, which would then be run locally
using each collaborators own resources. Finally, to achieve these
types of collaborative efforts, technologies to support these are
necessary; many of these are well established such as mailing lists
and chat systems, whereas a number are more recent such as service
registries. The following subsections discuss these facets of
collaborative computing in more detail.

\subsection{Shared Computing Services}

Sharing computing with web services ... SOAP, REST, XMPP~\cite{Wagener2009}.

\subsection{Sharing Computation Specifications}

Computation workflows can be shared with MyExperiment.org~\cite{Goble2010}.

\subsection{Collaboration Tools}

Taverna~\cite{Oinn2004} ... CDK-Taverna~\cite{Kuhn2010} ...

Bioclipse~\cite{Spjuth2009,Spjuth2007}...

\section{Collaborative Projects in Cheminformatics}

Facilitating communication ... mailing lists, blogs, ...

Collaborative documenting ... Google Docs, Google Wave~\cite{Neylon2009}, ...

\section{Conclusion}

\bibliographystyle{unsrt}
\bibliography{chapter}

\end{document}

